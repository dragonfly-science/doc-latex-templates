%We start by telling LaTeX that this is a dragonfly report. Note that this format
% is interchangeable with AEBR. Optionally we can say that
% it is a draft (which will add watermarks, line number and extra line spacing). 
\documentclass{doc-report}

% A report requires two titles. The first is the short title. This is used for the 
% footer text. The second is the full title. This goes on the front of the report. 
% You are only allowed to use two LaTeX macros in the title: \\ and \emph. If others
% are used there may be issues.
\title{Black petrel overlap with trawl and longline fisheries}{Overlap of the distribution of black petrel (\emph{Procellaria parkinsoni}) with  New Zealand trawl and longline fisheries}


% Reports also have a subtitle. This goes below the title on the front page.
\subtitle{Draft Final Research Report prepared for the Ministry for Primary Industries, as part of project SEA2013-06}

%Normally the inside cover citation uses the subtitle text at the end of the 
%citation. However this can be overriden using \citationnote
%\citationnote{A cool alternative}

% Authors must be included for a report. They should be listed with their full names
% with each author separated by \and.
\author{John Smith \and Jane E. Smith \and Jack Smith}

% In this example I have used the subcaption package to create a subfloat. These are
% when you have more than one picture within a single figure, and each gets it's own
% internal caption.
\usepackage{subcaption}

% The text on the back cover
% \backcover{%
% \textcopyright\  Copyright 2016, New Zealand Department of Conservation
% \bigskip\par
% Published by Publishing Team, Department of Conservation, PO Box 10420, The Terrace, Wellington 6143, New Zealand.
% \bigskip\par
% In the interest of forest conservation, we suggest paperless electronic publishing.}



% Dragonfly reports all have a licence attached. The default dragonfly licence can
% be used with \bydragonfly. The crown licence can be used with \bycrown. 
% You can also directly enter whichever licence is appropriate. 

% The lipsum package should not be used in a real report. It is used to generate
% filler text to test templates.
\usepackage{lipsum}

% This is the location of the bibliography file.
\addbibresource{test.bib}

% The actual document contents begin after here. 
\begin{document}

% Generates the title page
\maketitle

% Generates the table of contents
\tableofcontents

% Puts in an unnumbered section that shows up in the table of contents. If no 
% arguments are given, the section is called ``Executive Summary''.
\summary%[Custom Summary]

This has some boring text in it

\section{The First}

% Generate some fake text.
\lipsum[1]


\begin{figure}[h]
    \centering
  \includegraphics[width=0.9\textwidth]{birdsbyboat}
  \caption{This is a caption that is longer than one line. It is important to test 
  this behavior as this a common feature of the doc templates.}
\end{figure}

\subsection{Details}

% Generate fake text
\lipsum[2]


% Create a figure with subfigures inside of it. This uses the subcaption package
% rather than subfig
\begin{figure}[h]

  % Each subfigure need to know how wide it is. Otherwise it is the same as a normal
  % figure. 
  \begin{subfigure}{0.45\textwidth}
  \includegraphics[width=\textwidth]{wca}
  \caption{This is a caption that is longer than one line. It is important to test 
  this behavior as this a common feature of the doc templates.}
\end{subfigure}\qquad % it is nice to separate subfigures with a small space.
  \begin{subfigure}{0.45\textwidth}
  \includegraphics[width=\textwidth]{birdsbyboat}
  \caption{This is a caption that is longer than one line. It is important to test 
  this behavior as this a common feature of the doc templates.}
\end{subfigure}
% The overall figure also gets a caption
\caption{A figure of figures}
\end{figure}

\subsubsection{What's This?}
\lipsum[3] 
\paragraph{A minor heading}
\lipsum[5] 
\subparagraph{A really minor heading}
\lipsum[4]

Apa comes with a number of ways of citing things.
The interesting ones are:

% Here are a number of ways of citing documents. Note how cite, citet and citep
% are there as usual
\begin{itemize}
  \item \cite{abraham_summary_98-09}
  \item \citet{baker_nzclassification_2010}
  \item \citep{doc_sealion_2009}
  \item \parencite{gales_phocarctos_2008}
  \item \nptextcite{mpi_review_2012}
  %\item \fullcite{acap_summary_2011b}
  %\item \fullcitebib{acap_summary_2011b}
  \item \citeyear{robertson_population_2011}
  %\item \footcite{acap_summary_2011b}
  \item \textcite{roe_necropsy_2007}
  \item \cite{baker_marinemammals_2010}
  \item \cite{baker_census_2010}
\end{itemize}

% The document is finished, and we want the bibiography to start on a new page, 
% so we put in a page break.
\clearpage

% Print the bibliography.
\printbibliography

% The document is finished

\clearpage
\appendices

\section{}

\begin{figure}
  Some stuff in the figure
  \caption{A captain}
\end{figure}

\section{Another Appendix}

This just has some text in it.
\end{document}
