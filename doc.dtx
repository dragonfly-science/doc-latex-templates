% \iffalse meta-comment
% The MIT License (MIT)
% 
% Copyright (c) 2016 Department of Conservation
% 
% Permission is hereby granted, free of charge, to any person obtaining a copy
% of this software and associated documentation files (the "Software"), to deal
% in the Software without restriction, including without limitation the rights
% to use, copy, modify, merge, publish, distribute, sublicense, and/or sell
% copies of the Software, and to permit persons to whom the Software is
% furnished to do so, subject to the following conditions:
% 
% The above copyright notice and this permission notice shall be included in all
% copies or substantial portions of the Software.
% 
% THE SOFTWARE IS PROVIDED "AS IS", WITHOUT WARRANTY OF ANY KIND, EXPRESS OR
% IMPLIED, INCLUDING BUT NOT LIMITED TO THE WARRANTIES OF MERCHANTABILITY,
% FITNESS FOR A PARTICULAR PURPOSE AND NONINFRINGEMENT. IN NO EVENT SHALL THE
% AUTHORS OR COPYRIGHT HOLDERS BE LIABLE FOR ANY CLAIM, DAMAGES OR OTHER
% LIABILITY, WHETHER IN AN ACTION OF CONTRACT, TORT OR OTHERWISE, ARISING FROM,
% OUT OF OR IN CONNECTION WITH THE SOFTWARE OR THE USE OR OTHER DEALINGS IN THE
% SOFTWARE.
% \fi
%
% \iffalse
%<common|report|article|letter|beamer|proposal>\NeedsTeXFormat{LaTeX2e}[1999/12/01]
%
%<common>\ProvidesPackage{doc}[2016/06/29 v0.1 Common formatting requirements for Department of Conservation]
%<report>\ProvidesClass{doc-report}[2016/06/29 v0.1 Report format for Department of Conservation]
%<article>\ProvidesClass{doc-article}[2016/06/30 v0.1 Article format for Department of Conservation]
%<letter>\ProvidesClass{doc-letter}[2016/06/30 v0.5 Letter format for Department of Conservation]
%<beamer>\ProvidesClass{beamerthemeDragonfly}[2016/06/30 v0.1 Beamer format for Department of Conservation]
%<proposal>\ProvidesClass{doc-proposal}[2016/06/30 v0.1 Proposal format for Department of Conservation]
%<*driver>
\documentclass{ltxdoc}
\usepackage{doc}
\usepackage[toc]{multitoc}

\usepackage{tocloft}
\setcounter{tocdepth}{2}

\renewcommand{\contentsname}{CONTENTS}
\makeatletter
\newcommand\doc@pagefont{\fontsize{9pt}{13.2pt}\bfseries\sffamily\color{gray}}
\renewcommand{\cftsecpagefont}{\doc@pagefont}
\renewcommand{\cftsubsecpagefont}{\doc@pagefont}
\renewcommand{\cftsubsubsecpagefont}{\doc@pagefont}
\renewcommand{\cftdot}{-}
\renewcommand{\cftdotsep}{1}
\newcommand{\doc@leader}{\phantom{m}\color{gray}\tiny \cftdotfill{\cftdotsep}\phantom{m}}

\renewcommand{\cftsecleader}{\doc@leader}
\renewcommand{\cftsubsecleader}{\doc@leader}
\renewcommand{\cftsubsubsecleader}{\doc@leader}
\newcommand{\doc@tocfont}{\fontsize{9pt}{13.2pt}\bfseries\sffamily}

\renewcommand{\cftsecfont}{\doc@tocfont}
\renewcommand{\cftsubsecfont}{\doc@tocfont}
\renewcommand{\cftsubsubsecfont}{\doc@tocfont}
\setlength{\cftparskip}{1ex}
\setlength{\cftbeforesecskip}{2ex}

\makeatother
\usepackage{titlesec}

\titleformat{\section}{\color{alpine-tone}\fontsize{14pt}{16.8pt}\bfseries\neue\uppercase}{\thesection.}{1em}{}
\titleformat{\subsection}{\fontsize{11pt}{13.2pt}\bfseries\neue}{\thesubsection}{1em}{}
\titleformat{\subsubsection}{\fontsize{11pt}{13.2pt}\bfseries\neue}{\thesubsubsection}{1em}{}
\titlespacing{\section}{0pt}{\baselineskip}{0pt}
\titlespacing{\subsection}{0pt}{\baselineskip}{0pt}
\titlespacing{\subsubsection}{0pt}{\baselineskip}{0pt}
\EnableCrossrefs
\CodelineIndex
\RecordChanges
\begin{document}
  \DocInput{doc.dtx}
\end{document}
%</driver>
%\fi
%
% \CheckSum{0}
%
% \CharacterTable
%  {Upper-case    \A\B\C\D\E\F\G\H\I\J\K\L\M\N\O\P\Q\R\S\T\U\V\W\X\Y\Z
%   Lower-case    \a\b\c\d\e\f\g\h\i\j\k\l\m\n\o\p\q\r\s\t\u\v\w\x\y\z
%   Digits        \0\1\2\3\4\5\6\7\8\9
%   Exclamation   \!     Double quote  \"     Hash (number) \#
%   Dollar        \$     Percent       \%     Ampersand     \&
%   Acute accent  \'     Left paren    \(     Right paren   \)
%   Asterisk      \*     Plus          \+     Comma         \,
%   Minus         \-     Point         \.     Solidus       \/
%   Colon         \:     Semicolon     \;     Less than     \<
%   Equals        \=     Greater than  \>     Question mark \?
%   Commercial at \@     Left bracket  \[     Backslash     \\
%   Right bracket \]     Circumflex    \^     Underscore    \_
%   Grave accent  \`     Left brace    \{     Vertical bar  \|
%   Right brace   \}     Tilde         \~}
%
% \changes{0.1}{2016/06/30}{Initial Version}
%
% \GetFileInfo{doc.dtx}
%
% \title{The \textsf{Dragonfly} formatting package}
% \author{Department of Conservation}
% \maketitle
%
% \begin{abstract}
%  This is a set of classes that are used to format documents according the 
%  Department of Conservation formatting requirements. There are a range of different
%  document classes as well as a style file which provides all of the common 
%  implementation.
% \end{abstract}
% \clearpage
% \tableofcontents
%  \clearpage
% \section{Class and Package Options}
%
% All options are passed down to the common doc layer. 
%
% Since we intend on manually hiding certain options we need a command to 
% clear it from the global options list. This is taken from 
% \url{http://tex.stackexchange.com/questions/33245/disabling-the-draft-option-in-a-package}.
%    \begin{macrocode}
%<*letter|article|report|proposal|beamer>
\def\@clearglobaloption#1{%
  \def\@tempa{#1}%
  \def\@tempb{\@gobble}%
  \@for\next:=\@classoptionslist\do
    {\ifx\next\@tempa
       \message{Cleared  option \next\space from global list}%
     \else
       \edef\@tempb{\@tempb,\next}%
     \fi}%
  \let\@classoptionslist\@tempb
  \expandafter\ifx\@tempb\@gobble
    \let\@classoptionslist\@empty
  \fi}
%</letter|article|report|proposal|beamer>
%    \end{macrocode}
%
% The doc package needs to recognise the draft option, and just
% give it to ifdraft.
%
%    \begin{macrocode}
%<*common>
\DeclareOption{draft}{
  \PassOptionsToPackage{\CurrentOption}{ifdraft}
}
\ProcessOptions\relax
%</common>
%    \end{macrocode}
% 
% The rest are just default behaviour. 
%    \begin{macrocode}
%<*letter|article|report|proposal|beamer>
\DeclareOption{draft}{
  \PassOptionsToPackage{\CurrentOption}{doc}
  \@clearglobaloption{draft}
}
\ProcessOptions\relax
%</letter|article|report|proposal|beamer>
%    \end{macrocode}
%
% \section{Templates}
% Each different class is based on a different template. The following
% styles have classes implemented:
%
% \begin{itemize}
% \item Dragonfly report
% \item Dragonfly letter
% \item Dragonfly article
% \item Dragonfly letter
% \item Dragonfly beamer theme
% \item Aquatic Environment and Biodiversity Report
% \end{itemize}
%
% The basic structure is the same as the base latex article class
% with small changes to correct the 
% theme. The beamer theme is based on the default with inner theme circles,
% and the color theme using Dragonfly colours.
%
%    \begin{macrocode}
%<letter|article|report>\LoadClass[a4paper,final,11pt]{article}
%<proposal>\LoadClass[a4paper,11pt]{doc-report}
%<beamer>\mode<presentation>
%<beamer>\useinnertheme{circles}
%<beamer>\usecolortheme[RGB={28,152,196}]{structure}
%    \end{macrocode}
%
%
% All of the classese use a commmon style to share general definitions.
% This is provided by the doc style.
%    \begin{macrocode}
%<report|article|letter|beamer>\RequirePackage{doc}
%    \end{macrocode}
%
% \section{Page Layout}
%
% \subsection{Letter and Article}
% Letters and articles are very similar looking formats. They have the same margins
% which match what has been used historically. 
%    \begin{macrocode}
%<*letter|article>
\RequirePackage[
    hmargin={51mm,30mm},
    vmargin={26mm,40mm}]{geometry}
%</letter|article>
%    \end{macrocode}
%
% \subsection{Report}
%
% The report template has been created by a designer and has very strict 
% rules about exactly what all the margins are.
%    \begin{macrocode}
%<*report>
\RequirePackage[
  left=52mm,
  right=30mm,
  top=30mm,
bottom=40mm
]{geometry}
%</report>
%    \end{macrocode}
%
%
% \section{Fonts}
%
% The Department of Conservation reports in general use Archer and
% Omnes as their fonts.  \begin{macrocode}
%<*common>
\RequirePackage{fontspec}
\setmainfont[Mapping=tex-text,
    ItalicFont     = {Archer Book Italic},
    BoldFont       = {Archer Bold},
    BoldItalicFont = {Archer Bold Italic}]{Archer Book}

\newfontfamily\neue[Mapping=tex-text,
    Numbers=Monospaced,
    ItalicFont     = {HelveticaNeueLTStd-It},
    BoldFont       = {HelveticaNeueLTStd-Bd},
    BoldItalicFont = {HelveticaNeueLTStd-BdIt}]{HelveticaNeueLTStd-Roman}
    
\setsansfont[Mapping=tex-text,
    Numbers=Monospaced,
    ItalicFont     = {HelveticaNeueLTStd-It},
    BoldFont       = {HelveticaNeueLTStd-Bd},
    BoldItalicFont = {HelveticaNeueLTStd-BdIt}]{HelveticaNeueLTStd-Roman}

\setmathrm[Mapping=tex-text,
    Numbers=Monospaced,
    ItalicFont     = {HelveticaNeueLTStd-It},
    BoldFont       = {HelveticaNeueLTStd-Bd},
    BoldItalicFont = {HelveticaNeueLTStd-BdIt}]{HelveticaNeueLTStd-Roman}
\setmathsf[Mapping=tex-text,
    Numbers=Monospaced,
    ItalicFont     = {HelveticaNeueLTStd-It},
    BoldFont       = {HelveticaNeueLTStd-Bd},
    BoldItalicFont = {HelveticaNeueLTStd-BdIt}]{HelveticaNeueLTStd-Roman}
\setmathtt[Mapping=tex-text, 
    Numbers=Monospaced,
    ItalicFont     = {HelveticaNeueLTStd-It},
    BoldFont       = {HelveticaNeueLTStd-Bd},
    BoldItalicFont = {HelveticaNeueLTStd-BdIt}]{HelveticaNeueLTStd-Roman}
%</common>
%    \end{macrocode}
%
%
% We also set the fontsize to be 10pt. 
%
%    \begin{macrocode}
%<common>\renewcommand{\normalsize}{\fontsize{9.5pt}{11pt}\selectfont}
%    \end{macrocode}
%
% Suppress hyphenation, and try to keep the lines justified.
%  \url{http://tex.stackexchange.com/questions/31301/how-to-reduce-the-number-of-hyphenation}
%
%    \begin{macrocode}
%<*report|proposal>
\pretolerance=5000
\tolerance=2000 
\emergencystretch=10pt
%</report|proposal>
%    \end{macrocode}
%
%
% All formats except for reports need are ragged right.
%    \begin{macrocode}
%<letter|article|beamer|proposal>\raggedright
%    \end{macrocode}
%
% Paragraphs are not indented, but there is extra spacing between them. 
%    \begin{macrocode}
%<common>\setlength{\parskip}{3mm}
%<common>\setlength{\parindent}{0mm}
%    \end{macrocode}
%
% \section{Coloring}
% Dragonfly uses a set of common custom colours. These are defined as follows. 
%    \begin{macrocode}
%<*common>
\RequirePackage[table]{xcolor}
\definecolor{lake}{cmyk}{95,0,3,26}
\definecolor{alpine}{cmyk}{80,95,0,0}
\definecolor{sunset}{cmyk}{5,95,0,0}
\definecolor{fauna}{cmyk}{0,75,90,0}
\definecolor{flora}{cmyk}{50,0,100,0}
\definecolor{kiwi}{cmyk}{75,0,100,68}
\definecolor{honey}{cmyk}{0,0,100,30}
%% Hints
\definecolor{lake-hint}{cmyk}{29,0,1,8} 
\definecolor{alpine-hint}{cmyk}{24,29,0,0}
\definecolor{sunset-hint}{cmyk}{2,29,0,0}
\definecolor{fauna-hint}{cmyk}{0,22,27,0}
\definecolor{flora-hint}{cmyk}{15,0,30,0}
\definecolor{kiwi-hint}{cmyk}{0,22,30,20}
\definecolor{honey-hint}{cmyk}{0,0,30,9}
%% Tones
\definecolor{lake-tone}{cmyk}{95,0,3,50} 
\definecolor{alpine-tone}{cmyk}{80,95,0,50}
\definecolor{sunset-tone}{cmyk}{5,95,0,50}
\definecolor{fauna-tone}{cmyk}{0,75,90,50}
\definecolor{flora-tone}{cmyk}{50,0,100,50}
\definecolor{kiwi-tone}{cmyk}{0,75,100,85}
\definecolor{honey-tone}{cmyk}{0,18,100,50}
%</common>
%    \end{macrocode}
%
% \section{Headings}
%
% Some styles need to be able to customise their headings if needed.
% Notably titlesec breaks beamer, so should not be loaded in that case.
%    \begin{macrocode}
%<*common>
\@ifclassloaded{beamer}{}{\RequirePackage{titlesec}}
%</common>
%    \end{macrocode}
%
%
% \subsection{Report}
% We need to set the headings to be correct
%
%    \begin{macrocode}
%<*report>
\titleformat{\section}{\color{alpine-tone}\fontsize{14pt}{16.8pt}\bfseries\neue\uppercase}{\thesection.}{1em}{}
\titleformat{\subsection}{\fontsize{11pt}{13.2pt}\bfseries\neue}{\thesubsection}{1em}{}
\titleformat{\subsubsection}{\fontsize{11pt}{13.2pt}\bfseries\neue}{\thesubsubsection}{1em}{}
\titlespacing{\section}{0pt}{\baselineskip}{0pt}
\titlespacing{\subsection}{0pt}{\baselineskip}{0pt}
\titlespacing{\subsubsection}{0pt}{\baselineskip}{0pt}
%</report>
%    \end{macrocode}
%
%
%
% \subsection{Beamer}
%
%
%
% To start with, we want to make sure that frame titles are right alined, 
% bold, neue and logo blue.
%
%    \begin{macrocode}
%<*beamer>
\setbeamertemplate{frametitle}[default][right]

\setbeamerfont{frametitle}{family=\neue,series=\bfseries}
\setbeamercolor{frametitle}{fg=alpine-tone}

\setbeamerfont{framesubtitle}{family=\neue,series=\mdseries}
\setbeamercolor{framesubtitle}{fg=black}
%</beamer>
%    \end{macrocode}

%
% General slides all have to quote the section name after the title in the header box.
% If there is a frame subtitle then it is printed instead. 
%
%    \begin{macrocode}
%<*beamer>
\setbeamertemplate{frametitle}{
   \ifbeamercolorempty[bg]{frametitle}{}{\nointerlineskip}%
  \begin{beamercolorbox}[right,wd=\textwidth]{frametitle}
    \usebeamerfont{frametitle}%
    \vspace*{2ex}\par%
    \strut\insertframetitle\strut\par%
    {%
      \ifx\insertframesubtitle\@empty{\usebeamerfont{framesubtitle}\usebeamercolor[fg]{framesubtitle}\insertsection\strut\par}%
      \else%
      {\usebeamerfont{framesubtitle}\usebeamercolor[fg]{framesubtitle}\insertframesubtitle\strut\par}%
      \fi
    }%
    \vskip-1ex%
    \if@tempswa\else\vskip-.3cm\fi% set inside beamercolorbox... evil here...
  \end{beamercolorbox}% 
}
%</beamer>
%    \end{macrocode}
%
% \section{Header and Footer}
%
% Each document class has it's own settings for headers and footers.
% As a result every package should use fancyhdr
%    \begin{macrocode}
%<common>\usepackage{fancyhdr}
%    \end{macrocode}
%
% \subsection{Letter}
% There should be no header or footer for letters. 
%    \begin{macrocode}
%<letter>\pagestyle{empty}
%    \end{macrocode}
%
% \subsection{Report}
%
% Reports have a page number in the bottom left, and the short report title in the 
% bottom right. There are no head or foot rules. 
%    \begin{macrocode}
%<*report>
\fancypagestyle{plain}{
    \fancyhf{}
    \fancyfoot[L]{\color{alpine-tone}\small\neue\bf\thepage}
    \fancyfoot[R]{\color{alpine-tone}\small\neue\bf \footertitle}
    \renewcommand{\headrulewidth}{0pt} 
}
%</report>
%    \end{macrocode}
%
% \subsection{Beamer}
%
%    \subsubsection{Meta Data}
% The presentations require commands to set and store the values used for the 
% title and normal page footers. These are defined simply with short hand commands
% to assign a known variable. This is because the footers are not constatnt. Also
% note that the titlepage footer and general footers are different.
%
%    \begin{macrocode}
%<*beamer>
\newcommand{\@titlefooter}{}
\newcommand{\@pagefooter}{}
\newcommand{\titlefooter}{%
  \renewcommand{\@titlefooter}%
}
\newcommand{\pagefooter}[1]{%
  \renewcommand{\@pagefooter}{#1}  
}
%</beamer>
%    \end{macrocode}
%
%
% \subsubsection{Footline}
%
% The footer should contain the company logo on the left and a message on the right. 
% The message should be customizable according to the presentation using |\pagefooter|.
%
%    \begin{macrocode}
%<*beamer>
\setbeamertemplate{footline}{
\leavevmode
\begin{beamercolorbox}[left,ht=4ex,dp=4ex,leftskip=0.8cm,wd=0.5\paperwidth]{default text}
 \includegraphics[width=0.18\textwidth]{logo}
\end{beamercolorbox}%
\begin{beamercolorbox}[right,rightskip=2ex,ht=4ex,dp=4ex,wd=0.5\paperwidth]{default text}
             \hfill   \neue \@pagefooter   
\end{beamercolorbox}
}
%</beamer>
%    \end{macrocode}
%
%
% \section{Bibliography}
%
% Since we are using biblatex with apa we need to use polyglossia to tell
% biblatex which language we are using. Actually polyglossia doesn't support 
% british directly so we need a slight hack. We also need the bibliography to be
% a proper section
%
%    \begin{macrocode}
%<*report>
\RequirePackage{csquotes}
\RequirePackage{polyglossia}
\setdefaultlanguage[variant=newzealand]{english}
\RequirePackage[style=mfish,backend=biber]{biblatex}
\DeclareLanguageMapping{english}{english-mfish}
\defbibheading{bibliography}[References]{\section{#1}}
%</report>
%    \end{macrocode}
%
% We define some helper cite commands for familiarity. 
%
%    \begin{macrocode}
%<*report>
\newcommand{\citet}{\cite}
\newcommand{\citep}{\parencite}
%</report>
%    \end{macrocode}
%
%
% \section{Title page}
%
% Each kind of documents has a different kind of header. 
% Each one shall be addressed separately. Additionally,
% each title page section for a class will be split into two parts:
% the first defines new commands for entering metadata into the titlepage
% and the second defines the actual creation of the title page. 
%
%    \begin{macrocode}
%<*common>
\RequirePackage[absolute,overlay]{textpos}
\RequirePackage{etoolbox}
\RequirePackage{xparse}
\RequirePackage{hyphenat}
\RequirePackage{graphicx}
%</common>
%    \end{macrocode}
%
% \subsection{Letter}
%
% A letter must have a tile, author and greeting or else it will fail to compile. 
%
% The letterhead is printed on the first page without user intervention.
% There is no case where the letterhead should be ommited. 
%    \begin{macrocode}
%<letter>\AfterEndPreamble{\maketitle}
%    \end{macrocode}
%
% \subsubsection{Meta Data}
%
% The greeting macro sets the greeting of the document. This command is mandatory
% and must occur before |\begin{document}|. It takes a single arugment which is 
% the greeting line and saves it for use in the letter head. E.g |\greeting{Dear John}|.
% 
%    \begin{macrocode}
%<*letter>
\newcommand{\greeting}[1]{
  \renewcommand\@greeting{#1}
}
%</letter>
%    \end{macrocode}
%
% |\@greeting| contains the value of the greeting to write. This is to be used
% in printing the title page. This starts of with an error value since it is required.
%
%    \begin{macrocode}
%<*letter>
\newcommand{\@greeting}{\ClassError{doc-letter}
{No \noexpand\greeting given}
{Please insert a \noexpand\greeting before the \noexpand\begin{document} in your file.}
}
%</letter>
%    \end{macrocode}
%
% The preamble macro sets the  date and recipient address.
% This command is not mandatory, but must occur before |\begin{document}|. 
%
% The command takes two arugments, a date and an address.
% We need a little bit of indirection in order to be able to use 
% |\obeylines| to preserve line breaks (which is also why |\preameble|
% is point free. 
%
%    \begin{macrocode}
%<*letter>
\newcommand{\preamble}{\bgroup\obeylines\@preamble@}

\newcommand{\@preamble@}[2]{
  \gdef\@preamble{
#1

\vspace*{\parskip}
\begingroup\setlength{\parskip}{0pt}

#2
\endgroup
\vspace*{\parskip}
}
\egroup
}
%</letter>
%    \end{macrocode}
%
% |\@preamble| contains the address and date for use in the title.
%
% This starts of with an empty value since it is not required.
%    \begin{macrocode}
%<letter>\newcommand{\@preamble}{}
%    \end{macrocode}
%
%
% We also introduce a signature command. This is just a helper macro to insert space
% or a figure (signature) between the closing and the name. This is intended to be used
% for digital signatures, or leaving space for signing. This preserves linebreaks in 
% the arguments. As for |\preamble| this requires an indirect point free macro. 
%
%    \begin{macrocode}
%<*letter>
\newcommand{\signature}{\bgroup\obeylines\@signature}

\NewDocumentCommand{\@signature}{m o m}{
  \vspace*{\parskip}
  \vspace*{\parskip}
  \begingroup
  \setlength{\parskip}{0pt}
  #1

\IfNoValueTF{#2}{\vspace*{1.2cm}}{\vspace{1ex}\includegraphics[height=1.2cm]{#2}\vspace{1ex}}
  
  #3
  \endgroup
}
%</letter>
%    \end{macrocode}
%
%
% \subsubsection{Maketitle}
%    \begin{macrocode}
%<letter>\renewcommand\maketitle{
%    \end{macrocode}
%
% We want to set up the document such that the pdf has a sensible title and author set. 
%  
%    \begin{macrocode}
%<*letter>
    \hypersetup{  
      pdfinfo={  
        Title={\@title},  
        Author={\@author}
      }  
    }
%</letter>
%    \end{macrocode}
%
% We now need to place the company details in the top left hand corner.
%
%    \begin{macrocode}
%<*letter>   
    \begin{textblock*}{\textwidth}(35mm,15mm)
    \parbox{44mm}{\fontsize{6.5pt}{7.8pt}\selectfont\neue\noindent\rule{40mm}{0.1mm}
                  Level 5, 158 Victoria St\\
                  PO Box 27535, Wellington 6141 \\ 
                  New Zealand}
    \end{textblock*}
%</letter>
%    \end{macrocode}
% 
% We place the bottom of the details block separately from the top, as it has
% an additional specified location.

%    \begin{macrocode}
%<*letter>
    \begin{textblock*}{43mm}(51mm,26mm)
    \parbox{43mm}{\fontsize{6.5pt}{7.8pt}\selectfont\neue +64 4 385 9285 Phone\\
                  inbox@dragonfly.co.nz
                  \rule[0.5\baselineskip]{30mm}{0.1mm}
                  {\fontsize{7.5pt}{8pt}\selectfont\bf\color{alpine-tone}\neue www.dragonfly.co.nz}}
   \end{textblock*}
%</letter>
%    \end{macrocode}
%
% The logo is placed in the left margin, such that its right edge is touching the 
%  margin. 
%
%    \begin{macrocode}
%<*letter>   
    \begin{textblock*}{37mm}(14mm,47mm)
        \includegraphics[width=37mm]{logo}
    \end{textblock*}
%</letter>
%    \end{macrocode}
%
% This is all that goes in the letter head. At this point we clear all the commands that
% allow for setting header properties (as is standard in class files). We do not clear the
% title as we require it for the greeting. 
%
%    \begin{macrocode}
%<*letter>   
   \global\let\thanks\relax
  \global\let\maketitle\relax
  \global\let\@thanks\@empty
  \global\let\@author\@empty
  \global\let\@date\@empty
  \global\let\author\relax
  \global\let\date\relax
  \global\let\and\relax
%</letter>
%    \end{macrocode}
%
% We need to ensure that the text on the first page is moved down so that
% it starts 73mm down. Note that in order to make the space the right size, we need
% to subtract the \texttt{baselineskip} and \texttt{parskip}, as they will be 
% automatically added. 
%
%    \begin{macrocode}
%<*letter>   
   
  \vspace*{-\parskip}
  \vspace*{-\baselineskip}
  \vspace*{47mm}
%</letter>
%    \end{macrocode}
%
% At this point we add in the preamble if it has been declared, followed by the greeting.
%
%    \begin{macrocode}
%<*letter>   
\@preamble 

\@greeting

\textbf{\@title}

\global\let\title\relax
\global\let\greeting\relax
\global\let\preamble\relax
\global\let\@title\@empty 
\global\let\@greeting\@empty 
\global\let\@preamble\@empty 
}
%</letter>
%    \end{macrocode}
%
% The title block is now finished. 
%
% \subsection{Report, Proposal And AEBR}
%
% Report are a very strictly specified template the has a large number of fields.
% Note that since the proposal is a modified report, all these commands exist as well. 
% Additionaly, since the report template is subsititable for AEBR, all report commands
% should have an AEBR clone (that will often do nothing). This section will define the
% common metadata for both the report and AEBR formats.
%
% \subsubsection{Meta Data}
%
% The document has two titles. The first is the main title which goes on
% the cover of the document, while the second is a short title that goes 
% in the page footer. Both are required.  
%
%    \begin{macrocode}
%<*report>
\renewcommand\title[2]{%
  \renewcommand\@title{#2}%
  \renewcommand\footertitle{#1}%
}
\newcommand{\footertitle}{\ClassError{doc-report}{No Footer title defined}{Use \noexpand\title}}
%</report>
%    \end{macrocode}%
%
% Additionally to a full title, the report has a subtitle. This goes below the main
% title and is included in every document. In the AEBR this is ignored.
%    \begin{macrocode}
%<*report>
\newcommand{\extratitlepage}[1]{\renewcommand\@extratitlepage{#1}}
\newcommand{\@extratitlepage}{}
\newcommand{\subtitle}[1]{\renewcommand\@subtitle{#1}}
\newcommand{\@subtitle}{%
  \ClassError{doc-report}{No Subtitle Supplied}{Use subtitle}}
%</report>
%    \end{macrocode}
%
%
% Reports start with a large title image on the front page.
% The |\titleimage| macro allows users to specify the image to appear.
% Ideally the image should be at least 600dpi and ideally 210x110 mm.
% The image must be 210mm wide, and no more than 110 mm high. The default 
% Image is the wing pattern as a narrow band. This command is ignored in AEBR.
%
%    \begin{macrocode}
%<*report>
\newcommand{\@titleimage}{pattern}
\newcommand{\titleimage}[1]{%
  \renewcommand{\@titleimage}{#1}
}
%</report>
%    \end{macrocode}
%
% In the report the authors list starts with the title ``Authors:'',
% while in the proposal ``Prepared by:'':
%    \begin{macrocode}
%<report>\newcommand\@authorslabel{Authors}
%<proposal>\renewcommand\@authorslabel{Prepared by}
%    \end{macrocode}
%
% Like in standard document |\date| is allowed. However, it is only used for
% creating citations, and as such should contain only the year. As such the 
% default is changed from |\today| to the current year. In the AEBR format there
% is also a reportmonth command to set the month. 
%
%    \begin{macrocode}
%<*report>
  \renewcommand{\@date}{\the\year}
%</report>
%<report>\newcommand{\reportmonth}[1]{}
%    \end{macrocode}
%  
% Since the report title image is supposed to be a mixture of three images, we 
% may be required to cite the source of the images. This is also done on 
% the inside cover. We create three commands that take a licence image name
% and a source to cite the images. If no images are cited the block does not appear.
% In the AEBR these do nothing. 
%
%    \begin{macrocode}
%<*report>
\RequirePackage{lettrine}
\newcommand{\singleimage}[2]{%
\renewcommand{\@imagereferences}{Cover image:}
\renewcommand{\@singleimage}{\lettrine[lines=2,findent=6pt,nindent=0pt]{\includegraphics[height=1.8ex,keepaspectratio]{#1}}{}{\normalfont #2}}
}


\newcommand{\@singleimage}{}
\newcommand{\@leftimage}{}
\newcommand{\@middleimage}{}
\newcommand{\@rightimage}{}

\def\@imagereferences{}
%</report>
%
% Each report may contain a copyright notice. This has an image for the licence,
% as well as text. 
%
%    \begin{macrocode}
%<*report>
\newcommand{\licence}[1]{\renewcommand{\@licence}{#1}}
\newcommand{\@licence}{}
%</report>
%    \end{macrocode}
%
% AEBR documens require and ISSN, ISBN and report number. All of these must have 
% sensible defaults for before they are finished. These should not be implemented
% in report.
%    \begin{macrocode}
%<*report>
\newcommand{\issn}[1]{}
\newcommand{\isbn}[1]{}
\newcommand{\reportno}[1]{}
\newcommand{\reportseries}[1]{}
%</report>
%    \end{macrocode}
%
% The report also has a customisable citation note that goes at the end of the 
% citation. This defaults to the subtitle. 
%
%    \begin{macrocode}
%<*report>
\newcommand{\citationnote}[1]{\renewcommand{\@citationnote}{#1}}
\newcommand{\@citationnote}{\@subtitle}
%</report>
%    \end{macrocode}
%
% \subsection{Report and Propsal}
%
% Both the report and the title page share a similar first page, but a different second.
%
% \subsubsection{Maketitle}
%
% The title page for the report and proposal spans two pages. The first is cover page
% for the document, while the second is an inside flap that contains additional
% information.
%
%
% In order to lay out the title page we require a number of additional packages.
% These are for creating finding out the number of pages, 
% and adding in a background image. 
%
%    \begin{macrocode}
%<*report|proposal>
\RequirePackage{wallpaper}
%</report|proposal>
%    \end{macrocode}
%
% We start now describing the title page. This is also the title page for the 
% proposal as well. For simplicity, the proposal titlepage is the same as for
% the report, except that certain parts have been ommitted. 
%
% We start by redefining the maketitle command to custmize the title. 
%    \begin{macrocode}
%<report|proposal>\RenewDocumentCommand{\maketitle}{}{
%    \end{macrocode}
%
% The title infomation should also be used in the metadata of the PDF. 
%    \begin{macrocode}
%<*report|proposal>
{
\renewcommand{\and}{, }
\hypersetup{  
  pdfinfo={  
    Title={\footertitle},  
    Author={\@author}
  } 
} 
}
%</report|proposal>
%    \end{macrocode}
%
% The title page takes up a whole page which is blank and larger than usual.
%    \begin{macrocode}
%<*report|proposal>
\clearpage
\newgeometry{top=135mm,bottom=32mm,right=30mm,left=30mm}
\thispagestyle{empty}
%</report|proposal>
%    \end{macrocode}
%
% The title page starts with a full page with banner image 10mm from the top
% of the page. The rules around this image are described in detail where the 
% title image command is defined. 
%    \begin{macrocode}
%<*report|proposal>
\begin{textblock*}{\paperwidth}(0mm,10mm)
    % \includegraphics[width=\paperwidth]{\@titleimage}
\end{textblock*}
%</report|proposal>
%    \end{macrocode}
%
% The main text is the title of the document in large black neue. It also has some
% extra spacing following it.
%    \begin{macrocode}
%<*report|proposal>
{\raggedright \bf \neue \fontsize{23}{27.5}\selectfont \nohyphens{\@title}\par }
\vspace*{8mm}
\vspace*{-\baselineskip}
\vspace*{-\parskip}

%</report|proposal>
%    \end{macrocode}
%
% The main title is followed by the subtitle in alpine-tone.
%    \begin{macrocode}
%<report|proposal>{\raggedright \bf \color{alpine-tone} \fontsize{11pt}{13.5}\selectfont \nohyphens{\@subtitle}\par}
%    \end{macrocode}
%
%    If you want to put extra content in, you can use the extratitlepage command.
%    \begin{macrocode}
%<*report|proposal>
\vfill%
      \ifx\@extratitlepage\@empty%
      \else%
      {\@extratitlepage}%
      \fi%
\vfill%
\hspace*{90mm}%
\begin{minipage}[h]{65mm}
  \fontsize{7.5pt}{9pt}\selectfont\neue
  {
    \renewcommand{\and}{\\}
  \begin{tabular}{l}
  \textbf{\@authorslabel:}\\
    \@author
  \end{tabular}
  }
    \vspace{1mm}

    \noindent\rule{51mm}{0.1mm}

    \vspace{4mm}

    \hspace{15mm}
    \parbox{44mm}{\fontsize{7.5pt}{9pt}\selectfont PO Box 27535, Wellington 6141 \\ 
                 New Zealand \\ 
                 \href{www.dragonfly.co.nz}{\bfseries\color{alpine-tone}\neue dragonfly.co.nz}}
    
\end{minipage}
%</report|proposal>
%    \end{macrocode}
%    
% The Dragonfly logo is placed next to the author block. 
%    \begin{macrocode}
%<*report|proposal>
\begin{textblock*}{59mm}(68mm,253mm)
  % \includegraphics[width=\textwidth]{logo}
\end{textblock*}
%</report|proposal>
%    \end{macrocode}
%
% This finishes the title page, and now the inside cover begins.
% This is a normal size page with no footer and a patterned background. 
% All text on this page is aligned to the bottom. 
%    \begin{macrocode}
%<*report|proposal>

\restoregeometry
\clearpage
\thispagestyle{empty}

\vspace*{\fill}
%</report|proposal>
%    \end{macrocode}
%
%    \begin{macrocode}
%<*report>
{\neue \color{alpine-tone} \fontsize{11pt}{13.2pt}\selectfont \textbf{Cover Notes}}

To be cited as:

{

  \begin{refsection}[\jobname-self.bib]
  \renewcommand{\\}{}
  \renewcommand*{\multinamedelim}{;\space}
  \renewcommand*{\finalnamedelim}{;\space}%
  \renewcommand{\blx@maxcitenames}{25} %The number of authors here can be large
  \citename{this}[article]{author}\ (\citefield{this}{year}).\ \citefield{this}{title},\ \pageref{LastPage}\ pages.\ \ifdefempty{\@citationnote}{}{\@citationnote.}
  \end{refsection}
}

\@licence

\ifx\@imagereferences\@empty%
\relax%
\else%
\@imagereferences
\@singleimage
\fi

%</report>
%<*report|proposal>
\clearpage
\pagestyle{plain}
}
%</report|proposal>
%    \end{macrocode}
%
%
% \subsection{Beamer}
% Beamer title pages are similar to the default beamer ones, but we add some helper
% function and metadata.
%
%    \subsubsection{Meta Data}
%
% Dragonfly presentations often contain a project code. This is essentially the 
% equivalent of institute in a standard beamer document. We define project code
% to be an alias for institute. 
%
%    \begin{macrocode}
%<*beamer>
\newcommand{\projectcode}{\institute}
%</beamer>
%    \end{macrocode}
%
% \subsubsection{Titlepage}
%
% We now consider the appearance of the title page. First we set the relevant fonts.
%
%    \begin{macrocode}
%<*beamer>
\setbeamercolor{title}{fg=alpine-tone}
\setbeamerfont{title}{family=\neue,series=\bfseries,size={\fontsize{18}{20}}}
\setbeamercolor{subtitle}{fg=alpine-tone}
\setbeamerfont{subtitle}{parent=date,family=\neue, series=\bfseries}
\setbeamerfont{institute}{parent=date,size={\fontsize{10}{12}}}
%</beamer>
%    \end{macrocode}
%
% Every title page uses the Dragonfly logo at the bottom of the slide. 
%
%    \begin{macrocode}
%<beamer>\titlegraphic{} % \includegraphics[width=0.3\textwidth]{logo}}
%    \end{macrocode}
%
% We now describe the exact layout of the title page. 
%
%    \begin{macrocode}
%<*beamer>
\setbeamertemplate{title page}
{
%</beamer>
%    \end{macrocode}
%
% The top of the title page has a color banner accross the top using a standard
% doc pattern. 
%
%    \begin{macrocode}
%<*beamer>
  \vbox{}
  \vfill
  \begin{textblock*}{\paperwidth}(0mm,0mm)
    \includegraphics[width=\paperwidth]{pattern}
  \end{textblock*}
%</beamer>
%    \end{macrocode}
%
% The first thing of the page is the title of the presentation. This is followed by
% a space.
%
%    \begin{macrocode}
%<*beamer>
  \begin{centering}
    \begin{beamercolorbox}[sep=8pt,center]{title}
      \usebeamerfont{title}\inserttitle
    \end{beamercolorbox}%
      \ifx\insertsubtitle\@empty%
      \else%
        \vskip0.25em%
        {\usebeamerfont{subtitle}\usebeamercolor[fg]{subtitle}\insertsubtitle\par}%
      \fi%
    \vskip1em\par
%</beamer>
%    \end{macrocode}
%
% Following the title is the project code and author. The project code is not always present. It 
% is currently recorded as the institute to make use of beamers existing implmementation. It is probably
% necessary to include a |\projectcode| command later. This is followed immediately by the authors.
%
%    \begin{macrocode}
%<*beamer>
    \begin{beamercolorbox}[center]{author}
      \usebeamerfont{author}\insertauthor
    \end{beamercolorbox}
      \vspace{0.2cm}
      \usebeamerfont{date}\insertdate
      \par
    \begin{beamercolorbox}[center]{institute}
      \usebeamerfont{institute}\insertinstitute
    \end{beamercolorbox}  
      \par
      \vfill
%</beamer>
%    \end{macrocode}
% Finally the last element is the company logo.
%
%    \begin{macrocode}
%<*beamer>
      \vfill
    {\usebeamercolor[fg]{titlegraphic}\inserttitlegraphic\par}
  \end{centering}
}
%</beamer>
%    \end{macrocode}
%
%
% Finally, since the titleslide contains more than just the standard title,
% it also changes the way the page looks, we define a helper function to 
% insert a title slide.
%
%    \begin{macrocode}
%<*beamer>
\NewDocumentCommand{\titleslide}{}{
  \begingroup
\setbeamertemplate{footline}{
\begin{beamercolorbox}[ht=3ex,dp=1ex,leftskip=1.4cm,rightskip=1.4cm]{default text}
    \neue \@titlefooter \vspace*{\baselineskip}
  \end{beamercolorbox}
}
\begin{frame}
\maketitle
\end{frame}
\endgroup
}
%</beamer>
%    \end{macrocode}
%
% \section{Table of Contents}
%
% Reports requires a table of contents that is formatted in a very specific way.
%
%    \begin{macrocode}
%<report>\RequirePackage{tocloft}
%    \end{macrocode}
%
%
% Dection heading are always shown in capitals
% in the contents page. This involves rebinding contents line, so that when the 
% title is used to construct a hyperref url, it doesn't break the url. This is
% used in report and AEBR.
%
%    \begin{macrocode}
%<*report>
\let\doc@contentsline=\contentsline
\renewcommand\contentsline[4]{\doc@contentsline{#1}{\ifstrequal{#1}{section}{\MakeUppercase{#2}}{#2}}{#3}{#4}}
%</report>
%    \end{macrocode}
%
%\subsection{Report}
%  We want the table of contents to be on its own page. So we append 
% a clear instruction to the end of the writing command
%
%    \begin{macrocode}
%<report>\g@addto@macro\@cfttocfinish{\clearpage}
%    \end{macrocode}
%
% The title of the contents page must be CONTENTS in large, bold, blue neue.
%
%    \begin{macrocode}
%<*report>
\renewcommand{\cfttoctitlefont}{\color{alpine-tone}\fontsize{14pt}{16.8pt}\bfseries\sffamily}
\addto\captionsenglish{%
  \renewcommand{\contentsname}{CONTENTS}
}
%</report>
%    \end{macrocode}
%
% In this style the toc should only be created to depth 3.
%    \begin{macrocode}
%<report>\setcounter{tocdepth}{3}
%    \end{macrocode}
%
% The numbers for all three level of heading are drawn the same. In gray neue. 
%
%    \begin{macrocode}
%<*report>
\newcommand{\doc@pagefont}{\fontsize{11pt}{13.2pt}\bfseries\sffamily\color{gray}}
\renewcommand{\cftsecpagefont}{\doc@pagefont}
\renewcommand{\cftsubsecpagefont}{\doc@pagefont}
\renewcommand{\cftsubsubsecpagefont}{\doc@pagefont}
%</report>
%    \end{macrocode}
% 
% Similarly all the lines between the heading levels are the same. A line of tiny
% gray dashes.
%
%    \begin{macrocode}
%<*report>
\renewcommand{\cftdot}{-}
\renewcommand{\cftdotsep}{1}
\newcommand{\doc@leader}{\phantom{m}\color{gray}\tiny \cftdotfill{\cftdotsep}\phantom{m}}

\renewcommand{\cftsecleader}{\doc@leader}
\renewcommand{\cftsubsecleader}{\doc@leader}
\renewcommand{\cftsubsubsecleader}{\doc@leader}
%</report>
%    \end{macrocode}
%
%   All levels of heading have the same font for being written in. Bold, black neue. 
%
%    \begin{macrocode}
%<*report>
\newcommand{\doc@tocfont}{\fontsize{11pt}{13.2pt}\bfseries\sffamily}

\renewcommand{\cftsecfont}{\doc@tocfont}
\renewcommand{\cftsubsecfont}{\doc@tocfont}
\renewcommand{\cftsubsubsecfont}{\doc@tocfont}
%</report>
%    \end{macrocode}
%
% The spacing in the default table of contents is too small for this report format. 
% As a result, we increase the item spacing to make it less crowded.
%
%    \begin{macrocode}
%<*report>
\setlength{\cftparskip}{1ex}
\setlength{\cftbeforesecskip}{2ex}
%</report>
%    \end{macrocode}
%
%
% \section{Floats}
% 
% Some of the formats have strict rules concerning figures. These are implemented here.
% \subsection{Report}
%
% Reports need special figures and tables that go into the left margin. 
% To deal with this we will insert some commands into the figure command. 
% They also have rules about the formatting of the caption text. 
%
% We start by telling the caption that it should be wider.
%    
%    \begin{macrocode}
%<report>\RequirePackage[labelsep=colon,singlelinecheck=false,width=150mm]{caption}
%    \end{macrocode}
%
% The captions are written in the neue font, and with a bold Figure label. 
%
%    \begin{macrocode}
%<*report>
\renewcommand{\captionfont}{\small\neue}
\renewcommand{\captionlabelfont}{\small\bf\neue}
%</report>
%    \end{macrocode}
%
% We insert into the start and end of the figure command a 22mm indented minipage. 
%
%    \begin{macrocode}
%<*report>
\let\oldfigure\figure
\newcommand{\newfigurehead}{\hspace{-22mm}
  \addtolength{\textwidth}{22mm}
  \begin{minipage}[t]{\textwidth}}
\def\figure{\@ifnextchar[\figure@i \figure@ii}
\def\figure@i[#1]{\oldfigure[#1]\newfigurehead}
\def\figure@ii{\oldfigure\newfigurehead}

\let\endoldfigure\endfigure
\def\endfigure{\end{minipage}\endoldfigure}
%</report>
%    \end{macrocode}
%
% We then do the same thing for tables. 
%
%    \begin{macrocode}
%<*report>
\let\oldtable\table
\newcommand{\newtablehead}{\hspace{-22mm}
  \addtolength{\textwidth}{22mm}
  \begin{minipage}[t]{\textwidth}}
\def\table{\@ifnextchar[\table@i \table@ii}
\def\table@i[#1]{\oldtable[#1]\newtablehead}
\def\table@ii{\oldtable\newtablehead}

\let\endoldtable\endtable
\def\endtable{\end{minipage}\endoldtable}
%</report>
%    \end{macrocode}
%
%
%
% \section{Draft mode}
%
% Draft mode should create a watermark accross every page to make them as draft.
% It should also turn on double spacing and some line numbering for ease of editing.
% For clarity line numbers and double spacing will only start \emph{after} the title
% page.
%    \begin{macrocode}
%<*common>
\RequirePackage{setspace}
\RequirePackage{ifdraft}
\ifdraft{
\RequirePackage{draftwatermark}
\usepackage[modulo]{lineno}
\SetWatermarkColor{black!15}
\SetWatermarkFontSize{62pt}
\SetWatermarkText{\sffamily DRAFT - Not to be quoted}

\AtEndPreamble{
\g@addto@macro\maketitle{
\linenumbers
\doublespacing
}
}
}{}
%</common>
%    \end{macrocode}
%
%  \section{Additional Packages}
% The hyperref package must be the last one loaded, so we load it here.
% We also load booktabs as that is expected to be used for all tables. 
%
%    \begin{macrocode}
%<*common>
\RequirePackage{booktabs}
\RequirePackage{hyperref}
%</common>
%    \end{macrocode}
%
%    Hyperref should be used, but the links should not be coloured or have boxes around
% them. 
%    \begin{macrocode}
%<*common>
\hypersetup{
  colorlinks=true,
  urlcolor=black,
  citecolor=black,
  anchorcolor=black,
  pagecolor=black,
  menucolor=black,
  linkcolor=black
}
%</common>
%    \end{macrocode}
%
% \section{Self Citing}
%
% In order to automatically generate the citation using biblatex, we
% write out a generated bib file that contains an entry for the current document.
% This must happen at the end of the prelude, so that the title and so on are defined.
% Note that new lines have the be suppressed while doing this, and |\emph| commands
% need to print out the command rather than actually do emphasis. 
%
%    \begin{macrocode}
%<*report>
\RequirePackage{lastpage}
\AtBeginDocument{
{
\renewcommand{\and}{ and }
\renewcommand{\emph}[1]{\@backslashchar emph\@charlb #1\@charrb}
\renewcommand{\\}{}
\edef\text{@article\@charlb this,^^J%
options = \@charlb skipbib \@charrb,^^J%
year = \@charlb \@date \@charrb,^^J%
title = \@charlb \@title \@charrb,^^J%
author = \@charlb \@author \@charrb,^^J%
%</report>
%<*report>
\@charrb
}
\newwrite\tempfile
\immediate\openout\tempfile=\jobname-self.bib
\immediate\write\tempfile{\text}
\immediate\closeout\tempfile
}
}
%</report>
%    \end{macrocode}
%
%
% \section{Report}
%
%
% Reports contain an unnumbered summary section. We create a special command
% for putting this in.
%
%    \begin{macrocode}
%<*report>
\NewDocumentCommand{\summary}{O {Executive Summary}}{
  \phantomsection
  \addcontentsline{toc}{section}{\hspace{1.4em}#1}
  \section*{#1}
}
%</report>
%    \end{macrocode}
%
% \subsection{Creative commons reuse}
%
% Where possible, reports are licensed for re-use under a creative commons
% license. The copyright may be held by the crown or by Dragonfly. A creative
% commons attribution license is used.
%
%    \begin{macrocode}
%<*report>
\RequirePackage{lettrine}
\newcommand{\bydragonfly}{
  \lettrine[lines=2, findent=4pt]{\includegraphics[height=2ex]{by}}{}
  Copyright Dragonfly Data Science \textcopyright{} This report is licensed for re-use under a Creative
Commons Attribution 3.0 New Zealand Licence. This allows you to distribute,
use, and build upon this work, provided credit is given to the original source.
}
\newcommand{\bycrown}{
  \lettrine[lines=2]{\includegraphics[height=2ex]{by}}{}
  Crown copyright \textcopyright{} This report is licensed for re-use under a Creative
Commons Attribution 3.0 New Zealand Licence. This allows you to distribute,
use, and build upon this work, provided credit is given to the original source.
}
\newcommand{\ccby}[1]{
  \lettrine[lines=2]{\includegraphics[height=2ex]{by}}{}
  Copyright #1. This report is licensed for re-use under a Creative
Commons Attribution 3.0 New Zealand Licence. This allows you to distribute,
use, and build upon this work, provided credit is given to the original source.
}
%</report>
%    \end{macrocode}
%
%
% 
% \section{Beamer}
%
%
% We need to disable the navigational symbols put in the bottom right of the slides. 
%
%    \begin{macrocode}
%<beamer>\setbeamertemplate{navigation symbols}{}
%    \end{macrocode}
%
%
% We need a helper function to insert slides which are just a single image with no title
% or other text. Ideally the image should be at least a little bit centered. 
%
%    \begin{macrocode}
%<*beamer>
\NewDocumentCommand{\imageslide}{m O {w} o}{
  \begingroup
  \usebackgroundtemplate{
    \vbox to \paperheight{\vfil\hbox to \paperwidth{\hfil%
        \begingroup
        \if#2w%
           {\newcommand{\doc@imagewidth}{\paperwidth}
             \IfNoValueTF{#3}{\includegraphics[width=\doc@imagewidth]{#1}}%
                         {\includegraphics[width=\doc@imagewidth,#3]{#1}}}
           \else%
               {\newcommand{\doc@imageheight}{\paperheight}
                 \IfNoValueTF{#3}{\includegraphics[height=\doc@imageheight]{#1}}%
                             {\includegraphics[height=\doc@imageheight,#3]{#1}}}
               \fi%     
               \endgroup
               \hfil}\vfil}
  }
  \begin{frame}[plain]
  \end{frame}
  \endgroup
}
%</beamer>
%    \end{macrocode}
%
%The beamer template ends by stopping being specific to presentation
%    \begin{macrocode}
%<beamer>\mode<all>
%    \end{macrocode}
%
%
% \section{Appendicies}
%
%
% We require appendicies to be printed as ``Appendix X'' in the table of 
% contents. This is accomplished by using a the appendix package with the
% titletoc option. 
%    \begin{macrocode}
%<*report>
\RequirePackage[titletoc]{appendix}
%</report>
%    \end{macrocode}
%
% In the appendix we also wish to customize a number of properties. To do this
% we simply add these to the |\appendics| command.  These are:
% \begin{itemize}
%\item Section headings printed as Appendix X 
%\item Tables, equations and figures get the appendix letter before their number
%\item Counters are reset to set
% \end{itemize}
%    \begin{macrocode}
%<*report>
\g@addto@macro\appendices{
%</report>
%<report>\titleformat{\section}{\color{alpine-tone}\fontsize{14pt}{16.8pt}\bfseries\neue\uppercase}{APPENDIX \thesection }{1em}{}
%<*report>
\titleformat{\section}{\bfseries\sffamily}{APPENDIX \thesection }{1em}{}
\renewcommand{\theequation}{\thesection-\arabic{equation}}
\renewcommand{\thetable}{\thesection-\arabic{table}}
\renewcommand{\thefigure}{\thesection-\arabic{figure}}
\setcounter{section}{0}  % reset counter 
\setcounter{figure}{0}  % reset counter 
\setcounter{table}{0}  % reset counter 
\setcounter{equation}{0}  % reset counter 
}
%</report>
%    \end{macrocode}
\endinput
